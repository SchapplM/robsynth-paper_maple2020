% This is samplepaper.tex, a sample chapter demonstrating the
% LLNCS macro package for Springer Computer Science proceedings;
% Version 2.20 of 2017/10/04
%
\documentclass[runningheads]{llncs}
%
\usepackage{graphicx}
\usepackage{color}
\graphicspath{{../figures/}}
\usepackage[utf8]{inputenc}
% Used for displaying a sample figure. If possible, figure files should
% be included in EPS format.
%
% If you use the hyperref package, please uncomment the following line
% to display URLs in blue roman font according to Springer's eBook style:
\usepackage{url}
\renewcommand\UrlFont{\color{blue}\rmfamily}

\begin{document}
%
\title{A Maple Toolchain for Rigid Body Dynamics of Serial, Hybrid and Parallel Robots\thanks{Supported by German Research Foundation (DFG, grant OR 196/33-1), Federal Ministry of Education and Research of Germany (BMBF, grant 16SV6175) and European Union's Horizon 2020 research and innovation programme (grant 688857).}}
%
\titlerunning{Maple Toolchain for Robot Dynamics}
% If the paper title is too long for the running head, you can set
% an abbreviated paper title here
%
\author{Moritz Schappler\inst{1}\orcidID{0000-0001-7952-7363} \and \\
Tobias Ortmaier\inst{1}\orcidID{0000-0003-1644-3685}}
%
\authorrunning{M. Schappler et al.}
% First names are abbreviated in the running head.
% If there are more than two authors, 'et al.' is used.
%
\institute{
Institute of Mechatronic Systems, Leibniz University Hannover, Hannover, Germany\\
\email{\{moritz.schappler,tobias.ortmaier\}@uni-hannover.de}}
%
\maketitle              % typeset the header of the contribution
%
%\begin{abstract}
%The abstract should briefly summarize the contents of the paper in
%150--250 words.

%\keywords{Rigid body dynamics \and Robotics \and Symbolic code \and Toolchain.}
%\end{abstract}

\begin{itemize}
\item Die Gleichungen der Starrkörperdynamik von Robotern sollten zur schnellen Simulation von Systemen und für erweiterte Analysen in symbolischer Form aufgestellt werden.
\item Für einen systematischen Vergleich aller möglicher Roboterstrukturen für eine gegebene Aufgabe werden Datenbanken von hunderten Robotermodellen automatisch erzeugt.
\item Zur Integration in das wissenschaftliches Projekt ist eine Open-Source-Lösung für die Modellgenerierung wünschenswert, da eine geschlossene Lösung nicht erweiterbar ist. Ebenso ist ein hoher Automatisierungsgrad ohne manuelle Nacharbeit und eine direkte Testbarkeit der Ergebnisse wichtig.
\item Wir stellen eine neue Toolchain vor, mit der die Dynamik für verschiedene Typen von Roboter-Manipulatoren in symbolischer Form aufgestellt werden kann. Neben seriellen können hybride (Fig.\,\ref{fig:robot_examples}~a,b) und parallele Kinematiken (Fig.\,\ref{fig:robot_examples}~c) gelöst werden.
\item Die Implementierung in Maple und Bash-Skripte ist Open Source und hat das Ziel, Matlab-Funktionen zu generieren, mit denen weitere Analysen durchgeführt werden können. Der Code ist ebenfalls direkt in modellbasierten Regelungen basierend auf Simulink (Etherlab), SPS oder ROS einsetzbar.
\item Die Besonderheit der Toolbox gegenüber eines eher manuellen Prozesses ist die Erzeugung einer Stand-Alone-Testumgebung, bei der jede Funktion automatisch validiert wird. Dafür werden mathematische Zusammenhänge der Starrkörperdynamik genutzt (Energiekonsistenz, Implementierung verschiedener Verfahren wie Lagrange und Newton-Euler für serielle Ketten, D'Alembert und Lagrange für hybride Ketten)
\item Im Vortrag wird besonders auf die Implementierung der Toolchain eingegangen. Es gibt praktische Hinweise für die Nutzung von Maple mit einer Software-Versionsverwaltung zur Erreichung einer hohen Software-Qualität.
\item In der Zukunft kann die Toolchain auch zur Erzeugung von Testfällen für die symbolische Vereinfachung in Maple genutzt werden.
\end{itemize}

\begin{figure}[htb]
	\input{../figures/robot_examples_proposal.pdf_tex}
	\caption{Examples of different robots with dynamics equations in symbolic form.}
	\label{fig:robot_examples}
\end{figure} 

%
% ---- Bibliography ----
%
% BibTeX users should specify bibliography style 'splncs04'.
% References will then be sorted and formatted in the correct style.
%
%\bibliographystyle{splncs04}
%\bibliography{references}

\end{document}
