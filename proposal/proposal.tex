% !TEX encoding = UTF-8 Unicode
% !TEX spellcheck = en_US
%
\documentclass[runningheads]{llncs}
%
\usepackage{graphicx}
\usepackage{color}
\graphicspath{{../figures/}}
\usepackage[utf8]{inputenc}
% Used for displaying a sample figure. If possible, figure files should
% be included in EPS format.
%
% If you use the hyperref package, please uncomment the following line
% to display URLs in blue roman font according to Springer's eBook style:
\usepackage{url}
\renewcommand\UrlFont{\color{blue}\rmfamily}

\begin{document}
%
\title{A Maple Toolchain for Rigid Body Dynamics of Serial, Hybrid and Parallel Robots}%\thanks{Supported by German Research Foundation (DFG, grant OR 196/33-1), Federal Ministry of Education and Research of Germany (BMBF, grant 16SV6175) and European Union's Horizon 2020 research and innovation programme (grant 688857).}}
%
\titlerunning{Maple Toolchain for Robot Dynamics}
% If the paper title is too long for the running head, you can set
% an abbreviated paper title here
%
\author{Moritz Schappler\inst{1}\orcidID{0000-0001-7952-7363} \and
Tobias Ortmaier\inst{1}}%\orcidID{0000-0003-1644-3685}}

%
\authorrunning{M. Schappler et al.}
% First names are abbreviated in the running head.
% If there are more than two authors, 'et al.' is used.
%
\institute{
Institute of Mechatronic Systems, Leibniz University Hannover, Hannover, Germany\\
\email{\{moritz.schappler,tobias.ortmaier\}@uni-hannover.de}}
%
\maketitle              % typeset the header of the contribution
%
%\begin{abstract}
%The abstract should briefly summarize the contents of the paper in
%150--250 words.

%\keywords{Rigid body dynamics \and Robotics \and Symbolic code \and Toolchain.}
%\end{abstract}

%\begin{itemize}
%\item Die Gleichungen der Starrkörperdynamik von Robotern sollten zur schnellen Simulation von Systemen und für erweiterte Analysen in symbolischer Form aufgestellt werden.
The determination of the equations of motion in symbolic form is a common problem in robotics research.
It allows efficient simulations, model-based control and further analysis of a robot.
%Using the symbolic form of the rigid body dynamics enables fast forward and inverse dynamics simulations and allows further analysis of a system.
%\item Für einen systematischen Vergleich aller möglicher Roboterstrukturen für eine gegebene Aufgabe werden Datenbanken von hunderten Robotermodellen automatisch erzeugt.
A database with hundreds of robot models is created as an attempt to systematically compare all possible robot structures for a task, given by degrees of freedom (DoF) and an end effector trajectory.

Using their inverse dynamics models, the energy consumption of all robots is calculated
%With the models in the database necessary actuator forces or energy consumption of all robots is compared for the given trajectory
and the most efficient robot is recommended to the user.
%
Due to the high number of \textsc{Matlab} simulations necessary for a suitable comparison computational efficiency is the key to the success of the project.
The high number of robot models requires complete automation of the model generation and validation process.
Existing symbolic toolchains for dynamics model generation do not meet these requirements and necessitate manual processing of the results.
%\item Zur Integration in das wissenschaftliches Projekt ist eine Open-Source-Lösung für die Modellgenerierung wünschenswert, da eine geschlossene Lösung nicht erweiterbar ist. Ebenso ist ein hoher Automatisierungsgrad ohne manuelle Nacharbeit und eine direkte Testbarkeit der Ergebnisse wichtig.

%The toolchain for generating the dynamics models in the database should be editable (i.\,e. open source) to be able to test new features such as the exploitation of sparsity of parameter dependency to reduce computation time.
%Further a high automation without manual processing single models is favorable as well as high reliability of the results.
%\item Wir stellen eine neue Toolchain vor, mit der die Dynamik für verschiedene Typen von Roboter-Manipulatoren in symbolischer Form aufgestellt werden kann. Neben seriellen können hybride (Fig.\,\ref{fig:robot_examples}~a,b) und parallele Kinematiken (Fig.\,\ref{fig:robot_examples}~c) gelöst werden.

In this proposal, we present a fully-automated \textsc{Maple} toolchain for generating rigid body dynamics in symbolic form.
Serial (Fig.\,\ref{fig:robot_examples}\,a), hybrid (Fig.\,\ref{fig:robot_examples}\,b, c) and parallel kinematic structures (Fig.\,\ref{fig:robot_examples}\,d) can be obtained.
%\item Die Implementierung in Maple und Bash-Skripte ist Open Source und hat das Ziel, Matlab-Funktionen zu generieren, mit denen weitere Analysen durchgeführt werden können. Der Code ist ebenfalls direkt in modellbasierten Regelungen basierend auf Simulink (Etherlab), SPS oder ROS einsetzbar.

%\item Die Besonderheit der Toolbox gegenüber eines eher manuellen Prozesses ist die Erzeugung einer Stand-Alone-Testumgebung, bei der jede Funktion automatisch validiert wird. Dafür werden mathematische Zusammenhänge der Starrkörperdynamik genutzt (Energiekonsistenz, Implementierung verschiedener Verfahren wie Lagrange und Newton-Euler für serielle Ketten, D'Alembert und Lagrange für hybride Ketten)


\begin{figure}[h]
	\small % for caption in the figure
	\vspace{-0.6cm}
	\input{../figures/robot_examples_proposal.pdf_tex}
	\vspace{-0.7cm}
	\caption{Examples of different robots with dynamics equations in symbolic form.}
	\vspace{-0.5cm}
	\label{fig:robot_examples}
\end{figure} 

The symbolic computation is implemented with \textsc{Maple} worksheets in a modular way, using standard methods like Lagrange, Newton-Euler and D'Alembert.
The peculiarity lies in the framework of \textsc{Bash} scripts controlling the full workflow of the toolchain:
the optimized \textsc{Matlab} code generated by \textsc{Maple} is automatically converted to function files with proper documentation and input assertions.
A \textsc{Matlab/Simulink} unit test framework is automatically generated from template files and all functions for each robot are completely tested, rendering manual processing of the output unnecessary.
%\item Im Vortrag wird besonders auf die Implementierung der Toolchain eingegangen. Es gibt praktische Hinweise für die Nutzung von Maple mit einer Software-Versionsverwaltung zur Erreichung einer hohen Software-Qualität.
%\item In der Zukunft kann die Toolchain auch zur Erzeugung von Testfällen für die symbolische Vereinfachung in Maple genutzt werden.
%\end{itemize}
The presentation will highlight the implementation of the toolchain regarding the use of software version control (\texttt{git}) and achieving high quality by using unit tests.
In the future, the results may be used to provide test cases for improving the simplification algorithms in \textsc{Maple}.
%
The complete toolchain and various application examples are available under free license at 
\url{https://github.com/SchapplM/robsynth-modelgen}.
%
% ---- Bibliography ----
%
% BibTeX users should specify bibliography style 'splncs04'.
% References will then be sorted and formatted in the correct style.
%
%\bibliographystyle{splncs04}
%\bibliography{references}

\end{document}
