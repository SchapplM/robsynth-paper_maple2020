% !TEX encoding = UTF-8 Unicode
% !TEX spellcheck = en_US
%
\documentclass[runningheads]{llncs}
%
\usepackage{graphicx}
\graphicspath{{../figures/}}
\usepackage[utf8]{inputenc}
% Used for displaying a sample figure. If possible, figure files should
% be included in EPS format.
%
% If you use the hyperref package, please uncomment the following line
% to display URLs in blue roman font according to Springer's eBook style:
%\usepackage{url}
%\renewcommand\UrlFont{\color{blue}\rmfamily}

\begin{document}
%
\title{A Maple Toolchain for Rigid Body Dynamics of Serial, Hybrid and Parallel Robots}%\thanks{}}
%
%\titlerunning{Abbreviated paper title}
% If the paper title is too long for the running head, you can set
% an abbreviated paper title here
%
\author{Moritz Schappler\inst{1}\orcidID{0000-0001-7952-7363} \and Mark Wielitzka \and Tobias Ortmaier\inst{1}}%\orcidID{0000-0003-1644-3685}}
%
\authorrunning{M. Schappler et al.}
% First names are abbreviated in the running head.
% If there are more than two authors, 'et al.' is used.
%
\institute{
	Institute of Mechatronic Systems, Leibniz University Hannover, Germany\\
	\email{\{moritz.schappler,tobias.ortmaier\}@uni-hannover.de}}
%
\maketitle              % typeset the header of the contribution
%
\begin{abstract}
We present a fully-automated Maple toolchain for generating rigid body dynamics in symbolic form.
The peculiarity lies in the framework of Bash scripts controlling the full workflow of the toolchain: the optimized Matlab code generated by Maple is automatically converted to function files with proper documentation and input assertions.
A unit test framework is also generated.
All functions for the robot are completely tested, rendering manual post-processing of the output unnecessary. 

\keywords{Rigid body dynamics  \and Robotics \and Symbolic code \and Toolchain}
\end{abstract}
%
%
%
\section{Introduction}

\subsection{State of the Art}

Example references: \cite{KhalilCre1997,KhalilVijKhoMuk2014,SaminFis2013} and \cite{ZobovaHabVanDal2017}, \cite{BethgeMalTsaCal2017}, \cite{Docquier2013}. And why not \cite{FisettePosSasSam2002}, \cite{WangGonMunFis2019}, \cite{ShiMcP2000}, \cite{SousaCor2012}, \cite{Corke1998} and \cite{Robotran2020}?

\section{Symbolic Rigid Body Dynamics for Robots}

\section{Description of the Proposed Toolchain}

\section{Application for Model Database Framework}


\section{Conclusion}

\section*{Acknowledgements}

This work was developed using funding from the German Research Foundation (DFG, grant 341489206), the Federal Ministry of Education and Research of Germany (BMBF, grant 16SV6175) and the European Union's Horizon 2020 research and innovation programme (grant 688857).

% ---- Bibliography ----
\bibliographystyle{splncs04}
\bibliography{references}
\end{document}
